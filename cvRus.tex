% (c) 2002 Matthew Boedicker <mboedick@mboedick.org> (original author) http://mboedick.org
% (c) 2003-2007 David J. Grant <davidgrant-at-gmail.com> http://www.davidgrant.ca
% (c) 2008 Nathaniel Johnston <nathaniel@nathanieljohnston.com> http://www.nathanieljohnston.com
%
% (c) 2012 Scott Clark <sc932@cornell.edu> cam.cornell.edu/~sc932
%
%This work is licensed under the Creative Commons Attribution-Noncommercial-Share Alike 2.5 License. To view a copy of this license, visit http://creativecommons.org/licenses/by-nc-sa/2.5/ or send a letter to Creative Commons, 543 Howard Street, 5th Floor, San Francisco, California, 94105, USA.

\documentclass[letterpaper,11pt]{article}
\newlength{\outerbordwidth}
\pagestyle{empty}
\raggedbottom
\raggedright
\usepackage[svgnames]{xcolor}
\usepackage{framed}
\usepackage{tocloft}
\usepackage[T2A]{fontenc}
\usepackage[utf8]{inputenc}
\usepackage[english,russian]{babel}
\usepackage{tocloft}
\usepackage{etoolbox}
\usepackage[hidelinks]{hyperref} 
\robustify\cftdotfill
\usepackage{xcolor}
\usepackage{hyperref} 
\hypersetup{
	colorlinks=true,
	linkcolor=blue,
	filecolor=magenta,      
	urlcolor=cyan,
}


%-----------------------------------------------------------
%Edit these values as you see fit

\setlength{\outerbordwidth}{3pt}  % Width of border outside of title bars
\definecolor{shadecolor}{gray}{0.75}  % Outer background color of title bars (0 = black, 1 = white)
\definecolor{shadecolorB}{gray}{0.93}  % Inner background color of title bars


%-----------------------------------------------------------
%Margin setup

\setlength{\evensidemargin}{-0.25in}
\setlength{\headheight}{0in}
\setlength{\headsep}{0in}
\setlength{\oddsidemargin}{-0.25in}
\setlength{\paperheight}{11in}
\setlength{\paperwidth}{8.5in}
\setlength{\tabcolsep}{0in}
\setlength{\textheight}{9.5in}
\setlength{\textwidth}{7in}
\setlength{\topmargin}{-0.3in}
\setlength{\topskip}{0in}
\setlength{\voffset}{0.1in}


%-----------------------------------------------------------
%Custom commands
\newcommand{\resitem}[1]{\item #1 \vspace{-2pt}}
\newcommand{\resheading}[1]{\vspace{8pt}
  \parbox{\textwidth}{\setlength{\FrameSep}{\outerbordwidth}
    \begin{shaded}
\setlength{\fboxsep}{0pt}\framebox[\textwidth][l]{\setlength{\fboxsep}{4pt}\fcolorbox{shadecolorB}{shadecolorB}{\textbf{\sffamily{\mbox{~}\makebox[6.762in][l]{\large #1} \vphantom{p\^{E}}}}}}
    \end{shaded}
  }\vspace{-5pt}
}
\newcommand{\ressubheading}[4]{
\begin{tabular*}{6.5in}{l@{\cftdotfill{\cftsecdotsep}\extracolsep{\fill}}r}
		\textbf{#1} & #2 \\
		\textit{#3} & \textit{#4} \\
\end{tabular*}\vspace{-6pt}}
%-----------------------------------------------------------


\begin{document}

\begin{tabular*}{7in}{l@{\extracolsep{\fill}}r}
\textbf{\Large Самоделкина Мария} & \textbf{\today} \\
20 лет, Москва & E-mail: samodelkina.m.v@yandex.ru \\
& GitHub: \url{https://github.com/goo-goo-goo-joob} \\
& Tel: +7(985) 733-61-91 \\
\end{tabular*}
\\


%%%%%%%%%%%%%%%%%%%%%%%%%%%%%%
\resheading{Образование}
%%%%%%%%%%%%%%%%%%%%%%%%%%%%%%
\begin{itemize}

\item \ressubheading{НИУ ВШЭ, МИЭМ, Прикладная математика}{Москва, Россия}{Бакалавриат, 4 курс}{2017 - н.в.}

\begin{itemize}
	\resitem{Средний балл 9.53}
	\resitem{Первые места в рейтинге}
\end{itemize}

\item \ressubheading{Физико-математический лицей}{Глазов, Россия}{Среднее общее}{2006 - 2017}

\begin{itemize}
	\resitem{Аттестат с отличием}
\end{itemize}

\end{itemize}


%%%%%%%%%%%%%%%%%%%%%%%%%%%%%%
\resheading{Профессиональные навыки}
%%%%%%%%%%%%%%%%%%%%%%%%%%%%%%
\begin{itemize}
 \item Языки программирования: Python (pandas, numpy, scikit-learn, tensorflow, keras, catboost), Wolfram Mathematica, C++ %, C, Accembler
 \item Программы: LaTeX, Git, SQL, Unix %, MS Office
 \item Владения языками: English (upper intermediate)
\end{itemize}

%%%%%%%%%%%%%%%%%%%%%%%%%%%%%%
\resheading{Опыт}
%%%%%%%%%%%%%%%%%%%%%%%%%%%%%%
\begin{itemize}
	\item Стажер-аналитик в Тинькофф Банке \cftdotfill{\cftdotsep} 2020
	\item Стажер-исследователь  научно-учебной лаборатории телекоммуникационных систем МИЭМ НИУ ВШЭ \cftdotfill{\cftdotsep} 2018-2019
\end{itemize}

%%%%%%%%%%%%%%%%%%%%%%%%%%%%%%
%\resheading{Дополнительные курсы}
%%%%%%%%%%%%%%%%%%%%%%%%%%%%%%
%\begin{itemize}
%	\item Интеллектуальный анализ данных:
%\begin{itemize}
%	\resitem {Линейные модели, решающие деревья, оценка качества}
%	\resitem {Нейронные сети, глубокое обучение для изображений}
%	\resitem {Методы оптимизации, теория вероятностей, статистическое оценивание}
%\end{itemize}
%\end{itemize}

%%%%%%%%%%%%%%%%%%%%%%%%%%%%%%
\resheading{Проекты и достижения}
%%%%%%%%%%%%%%%%%%%%%%%%%%%%%%
\begin{itemize}
 \item \href{https://credit-risks.asciishell.ru/}{Программный модуль для управления кредитными рисками банка} - командный проект. Роль в проекте: тим-лидер, математическое исследование, программирование.
 \item \href{https://cups.mail.ru/results/leadersofdigital?page_size=18&period=past&round_id=583}{Всероссийский конкурс "Цифровой прорыв", задача от HeadHunter} о предсказании набора специализаций по контексту вакансии - 10 место.
 \item \href{https://yandex.ru/blog/dialogs/itogi-onlayn-khakatona-po-razrabotke-navykov-alisy-i-zapis-razbora-navykov}{Хакатон по разработке навыков Алисы} - призер. Разработан навык по проверке почты с помощью голосовых команд.
 \item \href{https://www.kaggle.com/c/chinese-char-recognition-smmo19/leaderboard}{Соревнование на Kaggle по распознаванию китайских иероглифов} - 2 место.
 \item \href{https://miem.hse.ru/armntk/winners2020}{Межвузовская научно-техническая конференция студентов, аспирантов и молодых специалистов имени Е.В.Арменского} - 3 место.
 
 
 %\item \href{https://github.com/goo-goo-goo-joob/IDA-HW/blob/master/hw4/homework_04_%D0%A1%D0%B0%D0%BC%D0%BE%D0%B4%D0%B5%D0%BB%D0%BA%D0%B8%D0%BD%D0%B0_%D0%9C%D0%B0%D1%80%D0%B8%D1%8F.ipynb}{Анализ кредитных рисков в Python} (определение платежеспособности клиента): обучение модели (градиентный бустинг), подбор оптимальных параметров алгоритмов, работа с категориальными признаками, определение важности признаков, визуализация.
%\item \href{https://github.com/goo-goo-goo-joob/IDA-HW/blob/master/hw3/homework_03_%D0%A1%D0%B0%D0%BC%D0%BE%D0%B4%D0%B5%D0%BB%D0%BA%D0%B8%D0%BD%D0%B0_%D0%9C%D0%B0%D1%80%D0%B8%D1%8F.ipynb}{Анализ текстовых данных в Python} (определение типа источника данных по тексту документа): обработка первичных данных, обучение модели (логистическая регрессия) и оценка результатов.	
 %\item Проведение исследования источника данных и предобработки данных: определение факторов, влияющих на уровень дохода, с помощью программы SAS Enterprice Miner; анализ взаимосвязей переменных посредством построения графиков; создание модели (дерева решений), описывающей поведение целевой переменной.
 %\item Легковесная криптография: изучение алгоритмов легковесной криптографии, анализ литературы, отчет "Хэш-функции в легковесной криптографии". 
\end{itemize}

%%%%%%%%%%%%%%%%%%%%%%%%%%%%%%
%\resheading{Дополнительная информация}
%%%%%%%%%%%%%%%%%%%%%%%%%%%%%%
%	\vspace{-2pt}
%	\begin{center}\begin{tabular*}{6.6in}{l@{\extracolsep{\fill}}r}
%		\multicolumn{2}{c}{Музыкальная школа № 1, Глазов (флейта, гитара) \cftdotfill{\cftdotsep} 2006-2016}\\
%       \multicolumn{2}{c}{The Mackenzie School of English, Edinburgh \cftdotfill{\cftdotsep} 2015, 2016}\\
%		\multicolumn{2}{c}{Волонтерская работа, Глазов, Москва \cftdotfill{\cftdotsep} 2016-2018}\\
%		\vphantom{E}
%\end{tabular*}
%\end{center}\vspace*{-16pt}


\end{document}
